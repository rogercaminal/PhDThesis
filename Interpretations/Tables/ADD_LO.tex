\begin{table}[ht!]
\begin{center}
\begin{footnotesize}
\begin{tabular}{c|c|ccccc}
\hline\hline
\multicolumn{7}{c} {\bf Limits on $\mathbf{M_D}$ [TeV]} \\
\hline
{\bf $\mathbf{n}$ extra-} & \multirow{2}{*}{\bf 95\% CL observed limit} & \multicolumn{5}{c}{{\bf 95\% CL expected limit}} \\
{\bf dimensions}          &        & $\mathbf{+2\sigma}$ & $\mathbf{+1\sigma}$ & {\bf Nominal} & $\mathbf{-1\sigma}$ & $\mathbf{-2\sigma}$ \\
\hline
2 & 5.25 & -0.77 & -0.42 & 5.33 & +0.45 & +0.88 \\
3 & 4.01 & -0.47 & -0.24 & 4.12 & +0.28 & +0.55 \\
4 & 3.53 & -0.35 & -0.18 & 3.65 & +0.19 & +0.40 \\
5 & 3.34 & -0.28 & -0.14 & 3.34 & +0.16 & +0.31 \\
6 & 3.07 & -0.24 & -0.12 & 3.21 & +0.12 & +0.26 \\
\hline\hline
\end{tabular}
\end{footnotesize}
\end{center}
\caption[The 95\% CL observed and expected limits on $M_D$ as a function of the number of extra-dimensions $n$ combining the most sensitive signal regions and considering LO signal cross sections.]{The 95\% CL observed and expected limits on $M_D$ as a function of the number of extra-dimensions $n$ combining the most sensitive signal regions and considering LO signal cross sections. The impact of the $\pm 1 \sigma$ theoretical uncertainty on the observed limits and the expected $\pm 1 \sigma$ range of limits in the absence of a signal are also given.}
\label{tab:ADD_Limits_LO}
\end{table}
