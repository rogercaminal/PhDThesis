\thispagestyle{empty}
\newpage
\pdfbookmark[0]{Abstract}{abstract} % Sets a PDF bookmark for the abstract
\chapter*{Abstract}

Results of a search for new phenomena are reported, using $\unit[20.3]{fb^{-1}}$ of proton-proton collision data at $\sqrt{s} = \unit[8]{TeV}$ recorded by the ATLAS experiment at the LHC.
A monojet analysis strategy is followed, in which events are required to have a very high energetic jet and large missing tranverse energy, with a maximum of three reconstructed jets.

No excess above the Standard Model background expectation is observed and the results are then interpreted several models.
On one hand, the monojet analysis allows to access a region in the parameter space of supersymmetric models not accessible for other analyses, since it has large sensitivity for compressed scenarios.
In particular, it is interpreted in terms of many models involving pair production of stops (with $\stoptocharm$ and $\stopfourbody$), sbottoms (with $\sbottomtob$), 1st and 2nd generation squarks (with $\squarktoq$) and gluinos (with $\gluinotobb$ and $\gluinotog$), while sensitivity studies are shown for the direct production of charginos or neutralinos ($\charginoneutralino$, $\charginochargino$ and $\squarkneutralino$).
Furthermore, an interpretation in terms of very light gravitino production in gauge mediated supersymmetry breaking scenarios is also performed.
On the other hand, other non-supersymmetric models beyond the Standard Model are also considered.
The monojet results are used to set limits on the fundamental Planck scale for ADD large extra dimension models.
Finally, limits on different effective theories and simplified models involving pair production of Weak Interacting Massive Particles (WIMPs) in colliders are also set.
These limits can be translated into WIMP-nucleon cross section limits and compared to direct searches experiments.
The results presented significantly extend previous results at colliders.

