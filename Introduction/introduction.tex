\chapter{Introduction}
    \label{chapter:Introduction}

The discovery of the Higgs boson in July 2012 by the ATLAS and CMS experiments, completes the Standard Model of particle physics.
Nonetheless, the description of the Universe with only the Standard Model (SM) is known to be incomplete.
The difficulty to model gravity in the same theoretical framework, the hierarchy problem, or the existence of Dark Matter are some of the many aspects of Nature that the SM cannot explain.

This Thesis presents a search for new phenomena in $\pp$ collisions at $\sqrt{s}=\unit[8]{TeV}$ recorded with the ATLAS detector at the LHC collider.
The final state under investigation is defined by the presence of a very energetic jet, large missing transverse energy, a maximum of three reconstructed jets, and no reconstructed leptons, leading to a monojet-like configuration.
The monojet final state constitutes a very clean and distinctive signature for new physics processes.

After the discovery of the Higgs and the constraints on the masses of first and second generation squarks and gluinos up to the TeV scale, much attention has been put to searches for third generation squarks.
These searches are motivated by naturalness arguments, which point to relatively light stops and sbottoms, and therefore allowing their production at the LHC.
The monojet analysis is interpreted in terms of pair production of stops and sbottoms, and in terms of inclusive searches for pair production of squarks, and gluinos.
In particular, this final state has large sensitivity to supersymmetric models involving a very compressed mass spectra of the superpartners in the final state (also known as ``compressed scenarios'').
This study is performed in the framework of complementing existing searches for squarks and gluinos that are not sensitive to such compressed spectra.

Monojet final states have been used traditionally to search for large extra dimensions and the production of Dark Matter (DM).
In this context, limits on the parameters of models involving the direct production of Kaluza-Klein towers of gravitons, neutralinos, or light gravitinos in gauge mediated supersymmetry breaking scenarios, are also considered.

This Thesis is organized as follows.
Chapter~\ref{chapter:StandardModel} provides an introduction to the SM theory, the QCD phenomenology at hadron colliders and the different Monte Carlo simulators used in the analysis.
Different scenarios for physics beyond the SM model are described in Chapter~\ref{chapter:BSM}.
Chapter~\ref{chapter:StatisticalModel} introduces the statistical model and the hypothesis testing that is used in the analysis.
The LHC collider and the ATLAS experiment are described in Chapter~\ref{chapter:ATLASDetector}.
Chapter~\ref{chapter:ReconstructionOfObjects} details the reconstruction of the different physics objects in ATLAS, and Chapter~\ref{chapter:MonojetAnalysis} describes the event selection, the background determination and the systematic uncertainties in full detail.
The final results and their interpretations in terms of the different models, are discussed from Chapters~\ref{chapter:Interpretations} to \ref{chapter:ADDGravitonProduction}.
Finally, Chapter~\ref{chapter:conclusions} is devoted to conclusions.
The document is complemented with several appendices.

The results presented in this thesis have led to the following publications by the ATLAS Collaboration:

\begin{itemize}
\item \emph{Search for pair-produced top squarks decaying into charm quarks and the lightest neutralinos using $\unit[20.3]{fb^{-1}}$ of $\pp$ collisions at $\sqrt{s} = \unit[8]{TeV}$ with the ATLAS detector at the LHC}, ATLAS-CONF-2013-068, \url{http://cds.cern.ch/record/1562880/}.

\item \emph{Search for pair-produced third-generation squarks decaying via charm quarks or in compressed supersymmetric scenarios in $pp$ collisions at $\sqrt{s}=8~$TeV with the ATLAS detector}, Phys. Rev. D90.052008, \href{http://www.arXiv.org/abs/arXiv:1407.0608/}{arXiv:1407.0608 [hep-ex]}.

\end{itemize}

The monojet results have also contributed to the summary notes of the searches for third generation squarks, and the searches for inclusive squarks and gluinos in Run~I, still not public by the time that this Thesis has been printed.
Furthermore, the interpretations of the monojet analysis in terms of large extra dimensions and the production of dark matter have significantly improved the previous ATLAS results, and are used to cross check the results from a new dedicated analysis in preparation.
