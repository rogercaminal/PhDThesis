\chapter{Conclusions}
    \label{chapter:conclusions}

This thesis presents results on the search for new phenomena using $\unit[20.3]{fb^{-1}}$ of proton-proton collision data at $\sqrt{s} = \unit[8]{TeV}$, recorded with the ATLAS experiment at the LHC.
Events with a very energetic jet, large missing transverse energy, a maximum of three reconstructed jets, and no reconstructed leptons are selected, leading to a monojet-like final state.
Six signal regions with thresholds on the leading jet $\pt$ and $\met$ ranging between 280~GeV and 600~GeV have been defined, in order to have sensitivity to a wide variety of models.

The Standard Model processes contributing to the monojet signal regions are dominated by the irreducible $\znn$, that accounts for more than 50\% of the total background.
The second most important background is the $\wtn$, which passes the signal region selection requirements when the $\tau$-leptons decay hadronically.
Contributions from $\wln$ and $\zll$+jets processes are also important, when the leptons are not reconstructed or are misreconstructed as jets.
The use of control regions, allows to extract the normalizations of the different $W/Z$+jets processes, and to significantly reduce the total systematic uncertainty, from 20\% to 30\%, to values between 3\% and 10\%, for the different signal regions.
Other minor backgrounds like the top and diboson contributions add up to about 6\% of the total background and are estimated directly from MC simulations.
The multijet and non-collision background contributions, negligible in most of the signal regions, are estimated with dedicated data-driven methods.

Good agreement is found between the data and the Standard Model background estimations.
The results are interpreted in terms of model independent 95\% confidence level (CL) upper limits on the visible cross section.
Values in the range between 96~fb and 5.2~fb are excluded for the different selections.

Exclusion limits at 95\% CL are set for models involving the direct pair production of third generation squarks in very compressed scenarios.
In particular, the pair production of stops with $\stoptocharm$ and/or $\stopfourbody$, and the pair production of sbottoms with $\sbottomtob$ are studied, leading to the exclusion of stop and sbottom masses below 260~GeV for very compressed scenarios.
Limits for direct production of first- and second-generation squarks or gluinos are also extracted in compressed scenarios, leading to exclusions for squarks and gluinos of 440~GeV and 600~GeV, respectively.
These limits extend the previous results from other dedicated searches.

The results of this analysis are also interpreted in terms of models in which Dark Matter (DM) candidates are directly produced.
Models involving Weakly Interacting Massive Particles (WIMPs), light gravitinos in Gauge Mediated Supersymmetric (GMSB) scenarios, or the direct production of electroweakinos, are studied.

For the pair production of WIMPs, an effective lagrangian is used to describe several types of interactions between WIMPs and SM particles, parametrized with different operators.
Exclusions on the WIMP-nucleon cross section are derived and compared to direct dark matter search experiments.
The ATLAS results give a unique access to WIMP masses $m_{\chi}<\unit[10]{GeV}$, where the direct detection suffers from kinematic suppression.
Simplified WIMP pair production models with the interaction among the SM particles and the WIMPs carried by a $Z'$ mediator are also considered.
For these models, limits on the mediator mass and the couplings of the model, are extracted.

The monojet results have also been interpreted in terms of associated production of gravitinos with a squark or gluino, for different configurations of the squark and gluino masses, $m_{\squark}$ and $m_{\gluino}$.
In the case of $m_{\squark} = m_{\gluino} = \unit[1]{TeV}$, gravitino masses below $\unit[4\times10^{-4}]{eV}$ can be excluded at 95\% CL.
This exclusion is then used to infer a lower bound on the scale of the supersymmetry breaking of $\sqrt{\langle F \rangle} \sim \unit[1]{TeV}$ at 95\% CL, significantly extending the previous limits from other experiments.

In the case of the direct production of charginos and neutralinos, $\charginoneutralino$, $\charginochargino$, and $\squarkneutralino$, the monojet analysis does not have enough sensitivity to exclude any parameter configuration involving $\ninotwo$, $\chinoonepm$, or $\squark$ masses larger than 100~GeV, respectively.

Finally, the monojet results are interpreted in terms of the ADD model of large extra dimensions.
The fundamental Plank scale in $4+n$ dimensions, $M_D$, is constrained, and values of $M_D$ below 5.8~TeV for $n=2$ and lower than 3.2~TeV for $n=6$ are excluded at 95\% CL, thus challenging the validity of a model that aims to solve the hierarchy problem.

In 2015, the LHC will resume the data-taking and provide $\pp$ collisions at 13/14~TeV, opening for a new energy frontier. 
In this new energy regime, the search for new phenomena in monojet final states will continue to play a central role in the ATLAS physics program.
